%\documentclass[a4paper,12pt]{article}%report
 
 \documentclass[a4paper,12pt]{article}%report
 % das hier ist eine font
 %\usepackage{newcent}
 
\usepackage{amsthm}
\usepackage[ngerman]{babel}     
\usepackage[utf8]{inputenc}      
\usepackage {amsmath}
\usepackage{bbm}
\usepackage{graphicx}
\usepackage{color}
\usepackage{hyperref}

\usepackage[arrow, matrix, curve]{xy}

%\newenvironment{ex}{begdef}{enddef}
% Theorem und Satz           
\newtheorem{thm}{Theorem}[section]
\newtheorem{myDef}[thm]{Definition}
% \newtheorem{thm}{Satz}[section]
\newtheorem{ex}{Aufgabe}[section]
% \newtheorem{myDef}[thm]{Definition}
\newtheorem{myExmpl}[thm]{Beispiel}
\newtheorem{myCmmt}[thm]{Bemerkung}
\newtheorem{lem}[thm]{Lemma}
\newtheorem{prop}[thm]{Proposition}
\newtheorem{myKor}[thm]{Korollar}
\newtheorem{myV}{Veranschaulichung}
\newtheorem{myLemma}[thm]{Lemma}

\usepackage[table,xcdraw]{xcolor}

\newcommand{\beg}[1]{\textsc{#1}}
\newcommand{\openEx}[1]{ {\bf Aufgabe #1:}}
\newcommand{\openAns}{ {\bf Antwort:~}}

% neue eigene shortcuts
\newcommand{\N}{\mathbbm N}
\newcommand{\K}{\mathbbm K}
\newcommand{\R}{\mathbbm R}
\newcommand{\A}{\mathcal{A}}
\newcommand{\B}{\mathcal{B}}
\newcommand{\F}{\mathcal{F}}
\newcommand{\U}{\mathcal{U}}
\newcommand{\PP}{\mathbbm P}
%\newcommand{\r}{\mathbbm R}
\newcommand{\C}{\mathbbm C}
\newcommand{\Z}{\mathbbm Z}
\newcommand{\zuZeigen}{ \mathrm{Z\kern-.5em\raise-0.4ex\hbox{Z}}}
\newcommand{\hnot}[1]{{\it #1}}
\newcommand{\Hnot}[1]{{\bf #1}}

\newcommand{\refFootnote}[1]{\footnote{#1}}
\definecolor{rahmen}{rgb}{.7,0.9,0.8}    
\definecolor{grund}{gray}{.99}           
\definecolor{schrift_box}{cmyk}{0.9,0.7,0.8,0}   
\definecolor{whites}{cmyk}{0, 0, 0,0}   
\definecolor{schrift_normal}{gray}{0}   
\definecolor{ekelig}{cmyk}{0.1,0.9,0.9,0}   
\definecolor{ekeligHintergrund}{cmyk}{0.1,0.1,0.9,0}   
  
  \newcommand{\defWord}[1]{{\rm #1}}
\newcommand{\openBox}[1]{
   \color{grund}
  
  \fcolorbox{rahmen}{grund}{\fbox{ \color{schrift_box}\parbox{\dimexpr \linewidth - 2\fboxrule - 2\fboxsep}{#1}}}
  \color{schrift_normal}}
  
\newcommand{\marker}[1]{
   \color{ekelig}
\fcolorbox{whites}{ekeligHintergrund}{
    TODO: #1
}
    \color{schrift_normal}}
  
\newcommand{\openBoxx}[2]{
   \color{grund}
  
  \fcolorbox{rahmen}{grund}{\fbox{ \color{schrift_box}\parbox{\dimexpr \linewidth - 2\fboxrule - 2\fboxsep}{#1}}{#2}}
  \color{schrift_normal}}
  
% Das hier ist nur so einnkommentiert, da ich auf OS X nicht herausgefunden 
% habe, wie man mathbbm zum Laufen kriegt.
%\newcommand{\N}{N}
%\newcommand{\K}{K}
%\newcommand{\R}{R}%
%\newcommand{\C}{C}
%\newcommand{\Z}{Z}
%\newcommand{\zuZeigen}{Z}

\newcommand*{\nametag}[1]{%
  \stepcounter{equation}%
  \tag{\tagtext{#1}\tagcomma\tagnumber{\theequation}}%
}
\makeatletter% <-- anklicken liefert Erklärung
\newcommand*{\eqnumref}[1]{%
  \begingroup
    \let\tagtext\@gobble
    \let\tagcomma\relax
    \eqref{#1}%
  \endgroup
}
\newcommand*{\eqtextref}[1]{%
  \begingroup
    \let\tagcomma\relax
    \let\tagnumber\@gobble
    \ref{#1}%
  \endgroup
} 


\newcommand{\grafik}[2]{\begin{figure}[!htb]
		\noindent\includegraphics[width=\linewidth,height=\textheight,
		keepaspectratio]{#1}
		\caption{\textrm{#2}}%
	\end{figure}}
\usepackage{cite}


\newcommand{\grafikM}[2]{\begin{figure}[!htb]
		\noindent\includegraphics[width=12.5cm,height=6.5cm]{#1}
		\caption{\textrm{#2}}%
	\end{figure}}
	\usepackage{cite}
	

\newcommand{\grafikMNon}[2]{\begin{figure}[!htb]
		\noindent\includegraphics[width=\linewidth]{#1}
		\caption{\textrm{#2}}%
	\end{figure}}
	\usepackage{cite}

\newcommand{\grafikMS}[2]{\begin{figure}[!htb]
		\noindent\includegraphics[height=4cm]{#1}
		\caption{\textrm{#2}}%
	\end{figure}}
	\usepackage{cite}
	

\newcommand{\grafikMHalf}[2]{\begin{figure}[!htb]
		\noindent\includegraphics[width=12.5cm,height=2.5cm]{#1}
		\caption{\textrm{#2}}%
	\end{figure}}

\usepackage{cite}




\newcommand{\newDef}[1]{ {\bf #1} }

\newcommand{\oldDef}[1]{ {\it #1} }




\usepackage{blindtext}

\usepackage{fancyhdr}%Kopf- und Fußzeile
\renewcommand{\headrulewidth}{1pt} %Linie oben
\fancyhf{}
\fancyhead[L]{\leftmark} %Kopfzeile links bzw. innen
\fancyhead[R]{\thepage} %Kopfzeile rechts bzw. außen


%Das hier ist neu dazugekommen für automatische Wörtertrennung.
\usepackage{xspace}



\setlength\parindent{0pt}




% zum abaendern wenn die Erklaerungen ausgeblendet werden sollen
\newcommand{\invisible}[1]{#1}
% \newcommand{\erklaerung}[1]{}
%opening
\title{Mitschrift Stochastik - Kapitel 2, Statistische Standardmodelle}
\author{Sarah, Julius}
	\pagestyle{fancy}
\begin{document}
	\maketitle
	\tableofcontents
	\section{Einführung}
	\newpage
	\section{Stochastische Standardmodelle}
	\subsection{Gleichverteilung }
	
	\begin{myDef}[diskrete Gleichverteilung]
		
		Wahrscheinlichkeitsraum $(\Omega, A, \PP)$ wird als Laplace-Raum bezeichnet.
		
		\begin{align*}
		n &:= | \Omega |, \\
		\\
		\Omega &:= \{\omega_1, \dots, \omega_n\}, \\
		\A &:= P(\Omega),\\
		\PP(A) &:= \frac{|A|}{|\Omega|} =: \U_\Omega
		\end{align*}
		{\bf Anwendung} wenn diskret und alle $\omega \in \Omega$ gleichberechtigt.
		
	\end{myDef}
	
	\invisible{
		
		Beispiel: Bose Einstein Verteilung
		System $n$ unterschiedlicher Teilchen, die sich in $N$ unterschiedlcihen Zellen befinden. 
		
		Suche die Anzahl der Teilchen in einer bestimmten Zelle.
		}
	
	
	
	
	
	
	
	\begin{myDef}[stetige Gleichverteilung, GLV in Kontinuum]
		Analog:
		\begin{align*}
			\Omega &:\subseteq R^n,\\
			\A &:= \B_{\Omega},\\
			\rho(x) &:= \frac{1}{\lambda^n(\Omega)},\\ 
			\U_\Omega(A) &:= \int \limits_{A} \rho (u) d u = \frac{\lambda(A)}{\lambda^n(\Omega)}
		\end{align*}
		{\bf Anwendung} wenn $\Omega$ nicht endlich und alle $\omega \in \Omega$ gleichberechtigt.
	\end{myDef}
	
	\newpage
	
	\subsection{Urnenmodell mit Zurücklegen }
	\subsubsection{geordnete Stichprobe}
	\begin{myDef}[geordentes Urnenmodell mit Zurücklegen]
		Es seien gegeben
		\begin{align*}
			N &:= \text{Anzahl Kugeln}, \\
			E &:= \text{Menge der Farben, hier soll } 2\leq |E| < \infty, \\
			a &:= \text{Farbe } a \in E,\\
			F_a &:= \text{Menge der Nummern der Kugeln mit Farbe } a \in E,\\
			N_a &:= |F_a|, \text{Anzahl der Kugeln der Farbe } a \in E,\\
			n&:= \text{Anzahl der Stichproben (Züge aus der Urne)}, \\
	\\
	\Omega&:=E^n, \\
	 \F&:=P(\Omega),\\
	 \PP &:= \text{ gesuchtes Wahrscheinlichkeitsmaß}.
		\end{align*}
		Zur Konstruktion des Maßes nummeriere die Kugeln mit Zahlen aus $\{1,\dots N \}$ durch und vergrößere künstlich die Beobachtungstiefe, sodass die bereits definierte diskrete Gleichverteilung verwendet werden kann.
		\begin{align*}
			\bar \Omega:=\{1, \dots, N \}^n, \quad
			\bar \F:=\PP(\bar \Omega), 			\quad
		\bar	\PP := \U_{\bar \Omega}.
		\end{align*}
		
		Erhalte somit durch Konstruktion einer Zufallsvariable $X : \bar \Omega \rightarrow \Omega$ durch Komponentenweise Betrachtung:
		$$ \PP(\{\omega\}\}) := \bar \PP \circ X^{-1} (\{\omega \}) = \prod\limits_{i = 1}^n \rho(\omega_i) $$
		mit 
		$$ \rho(\omega_i) := \frac{|N_{\omega_i}|}{N},$$
		in Worten
		Die Wahrscheinlichkeit, dass unter Berücksichtigung der Reihenfolge die Kugeln $\omega_1, \dots, \omega_n$  gezogen werden, ist gegeben durch
		$$ \prod\limits_{i = 1}^n\frac{\text{Anzahl der Kugeln Mit Farbe }\omega_i}{\text{Anzahl der Kugeln insgesamt}}.$$
	\end{myDef}
	
	\begin{myDef}
		Es sei $\rho$ Zähldichte auf $E$. Die {\bf $n-$fache Produktdichte} von $\rho$ ist definiert als $$\rho^{\times n} (\omega) := \prod \limits_{i = 1}^{n} \rho(\omega_i).$$
		Das zugehörige Wahrscheinlichkeitsmaß heißt {\bf $n-$faches Produktmaß} zu $\rho$.
	\end{myDef}
	\newpage
	\subsubsection{ungeordnete Stichprobe}
	Es ist wieder eine künstliche Vergrößerung der Beobachtungstiefe erforderlich. Definiere für $k_a$ Anzahl der Kugeln mit Farbe $a$ folgenden Wahrscheinlichkeitsraum:
	
	\begin{align*}
		\hat \Omega &:= \{\vec  k = (k_a)_{a\in E} : k_a \in \N_0, \sum\limits_{a \in E} k_a = n \}\\
		\hat \F &:= P(\Omega)		\\
		\hat \PP (\{\omega\}) &:= \binom{n}{\vec k} \cdot \prod_{a \in E} \rho(a)^{k_a}
	\end{align*}
	
	\begin{myDef}[Multinoumialkoeffizient] Für $\vec k = (k_a)_{a\in E}$ definiere
		$$ \binom{n}{\vec k} \cdot \prod_{a \in E} := \begin{cases}
		\frac{n!}{\prod\limits_{a\in E} k_a!} & \text{ falls } \sum\limits_{a\in E} k_a = n,\\
		0 & sonst.
		\end{cases}$$
	\end{myDef}
	
	
	\subsubsection{Zusammenfassung}
	$$\begin{xy}
		\xymatrix{
			Gleichverteilung   							&   (\bar\Omega, \bar\F, \bar\PP = \U_{\bar \Omega})  \ar[d]^{\text{reduktion Beobachtungstiefe }X}  \\
			geordnete Sticprobe   					 &  (\Omega, \F, \PP)  \ar[d]^{\text{reduktion Beobachtungstiefe }S}   \\
			unegordnete Stichprobe                &   (\hat\Omega, \hat\F, \hat\PP)  
		}
	\end{xy}$$
	Zugehörige Wahrscheinlichkeitsmaße:
	\begin{align*}
		 \PP(\{\omega\}) &:= \prod\limits_{i = 1}^n \rho(\omega_i) \\
		\hat \PP (\{\omega\}) &:= \binom{n}{\vec k} \cdot \prod_{a \in E} \rho(a)^{k_a}
	\end{align*}
	
	\subsection{Urnenmodell ohne Zurücklegen}
	\subsubsection{nummerierte Kugeln}
	\subsubsection{gefärbte Kugeln}
	\subsection{Poisson-Verteilung}
	\subsection{Wartezeitverteilung}
	\subsubsection{negative Binominialverteilung}
	\subsubsection{Gamma-Verteilung}
	\subsubsection{Die Beta-Verteilung}
	\subsection{Normalverteilung}
	\newpage
	\section{Bedingte Wahrscheinlichkeiten und Unabhängigkeit}
	\subsection{Bedingte Wahrscheinlichkeiten}
	%\subsection{Unabhängigkeit}
	
\end{document}